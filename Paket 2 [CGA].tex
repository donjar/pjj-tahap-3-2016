\documentclass{article}
\usepackage{amsmath,amsthm,amssymb,amsfonts,fullpage}
\pagenumbering{gobble}
\begin{document}
	\begin{center}
		{\huge PJJ Tahap 3 IMO 2016 Tidak Resmi}
		
		\vspace{1em}
		
		{\Large Paket 2}
		
		\vspace{1em}
		
		DJ Ilhan rmx
	\end{center}
	\section{Aturan Main}
	\begin{itemize}
		\item Solusi dikumpul paling lambat \textbf{Sabtu, 2 April 2016 22:59 WIB}.
		\item Soal dibuat sehingga no. 1 kira-kira sesusah no. 2 IMO, no. 3 sesusah no. 3 IMO. No. 2 somewhere in between. Anak tahap 3 udah ga jaman ngerjain soal no. 1 atau 4.
		\item Solusi boleh diketik, discan ataupun difoto.
		\item Kirimkan ke: 7744han@gmail.com ATAU h.i@u.nus.edu ATAU e0008984@u.nus.edu. Tulis nama Anda di subjek email.
		\item Umpan balik akan diberikan ke Anda. (Udah kayak KTO aja)
		\item Juara 1, 2, dan 3 bakal dapat hadiah (woo!)
		\item Soal akan di-post di olimpiade.org setelah batas pengumpulan. Harap partisipasinya dalam meramekan forum anak bangsa.
		\item Above all, PJJ ini tidak di-enforce kok. Iseng-iseng aja.
	\end{itemize}
	\section{Sorunlar}
	(Bahasa Turki. Sukses Afif ujiannya!)
	\vspace{1em}
	\begin{enumerate}
		\item Diberikan bilangan genap positif $n$; dua kotak di papan catur disebut bersebelahan bila mereka memiliki sisi persekutuan. Saya ingin menaruh pion-pion di petak-petak yang berbeda di papan catur $n \times n$ sehingga setiap petak (bisa jadi ada pion, bisa jadi tidak) bersebelahan dengan minimal satu petak yang ada pionnya. Setiap petak bisa ditaruh 1 atau lebih pion (bisa saja 10 miliyar). Cari banyaknya pion minimal yang harus ditaruh.
		\item Diberikan segitiga $ABC$; titik-titik $A'$, $B'$, dan $C'$ masing-masing adalah titik tengah segmen-segmen $BC$, $CA$, dan $AB$. Diberikan $P$ dan $P'$ sehingga $PA = P'A'$, $PB = P'B'$ dan $PC = P'C'$. Buktikan semua garis $PP'$ yang mungkin melewati sebuah titik tetap.
		\item Diberikan sebuah bilangan bulat positif $n$ dan bilangan-bilangan real $x_1, x_2, \dotsc, x_n$ dan $y_1, y_2, \dotsc, y_n$ sehingga untuk setiap $i, j \in \mathbb{N}$ yang memenuhi $i \le j \le n$ dipunyai $x_i \le x_j$ dan $y_i \ge y_j$, dan
			\[ \sum_{i=1}^n ix_i = \sum_{i=1}^n iy_i. \]
		Buktikan bahwa untuk setiap bilangan real $z$ dipunyai \[ \sum_{i=1}^n x_i \lfloor zi \rfloor \ge \sum_{i=1}^n y_i \lfloor zi \rfloor. \]
	\end{enumerate}
	\begin{center}
		\textit{``Saya menang. Karena saya tidak pernah kalah.'' - Johan Gunardi}
		
		\vspace{1em}
		
		\textit{(btw, saya kalah)}
	\end{center}
\end{document}