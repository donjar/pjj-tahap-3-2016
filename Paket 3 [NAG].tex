\documentclass{article}
\usepackage{amsmath,amsthm,amssymb,amsfonts,fullpage}
\pagenumbering{gobble}
\usepackage[utf8]{inputenc}
\usepackage[russian]{babel}
\begin{document}
	\begin{center}
		{\huge PJJ Tahap 3 IMO 2016 Tidak Resmi}
		
		\vspace{1em}
		
		{\Large Paket 3}
		
		\vspace{1em}
		
		DJ Ilhan rmx
	\end{center}
	\section{Aturan Main}
	\begin{itemize}
		\item Solusi dikumpul paling lambat \textbf{Sabtu, 9 April 2016 22:59 WIB}.
		\item Soal dibuat sehingga no. 1 kira-kira sesusah no. 2 IMO, no. 3 sesusah no. 3 IMO. No. 2 somewhere in between. Anak tahap 3 udah ga jaman ngerjain soal no. 1 atau 4.
		\item Solusi boleh diketik, discan ataupun difoto.
		\item Kirimkan ke: 7744han@gmail.com ATAU h.i@u.nus.edu ATAU e0008984@u.nus.edu. Tulis nama Anda di subjek email.
		\item Umpan balik akan diberikan ke Anda. (Udah kayak KTO aja)
		\item Juara 1, 2, dan 3 bakal dapat hadiah (woo!)
		\item Soal akan di-post di olimpiade.org setelah batas pengumpulan. Harap partisipasinya dalam meramekan forum anak bangsa.
		\item Above all, PJJ ini tidak di-enforce kok. Iseng-iseng aja.
	\end{itemize}
	\section{вопросов}
	(Bahasa Rusia, sebab Rusia adalah saingan utama Indonesia sejak dulu, dan tahun ini Indonesia akan mengalahkan Rusia)
	\vspace{1em}
	\begin{enumerate}
		\item Definisikan sebuah barisan $\{ a_n \}$ dimana \[ a_n = \frac{1}{n} \left( \left\lfloor \frac{n}{1} \right\rfloor + \left\lfloor \frac{n}{2} \right\rfloor + \dotsb + \left\lfloor \frac{n}{n} \right\rfloor \right). \] Buktikan terdapat tak berhingga banyaknya $n$ sehingga $a_{n + 1} > a_n$, dan juga terdapat tak berhingga banyaknya $n$ sehingga $a_{n + 1} < a_n$.
		\item Cari semua fungsi $f$ yang memetakan semua bilangan real ke bilangan real nonnegatif, sehingga untuk setiap $a, b, c, d \in \mathbb{R}$ yang memenuhi $ab + bc + cd = 0$, dipunyai \[ f(a - b) + f(c - d) = f(a) + f(b + c) + f(d).\]
		\item Misalkan $A_1A_2\dotso A_8$ adalah sebuah segi-8 konveks sama sisi sehingga $A_iA_{i+1}$ sejajar $A_{i+4}A_{i+5}$ untuk setiap $i$. Definisikan $B_i$ sebagai perpotongan $A_iA_{i+4}$ dan $A_{i-1}A_{i+1}$. Tunjukkan terdapat $i$ sehingga \[\frac{A_iA_{i+4}}{B_iB_{i+4}} \le \frac{3}{2}.\] ($8 | i - j$ mengimplikasikan $A_i = A_j$.)
	\end{enumerate}
	\begin{center}
		\textit{``Kapucino itu minuman paling rasis'' - Tobi Moektijono}
	\end{center}
\end{document}