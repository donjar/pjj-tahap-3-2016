\documentclass{article}
\usepackage{amsmath,amsthm,amssymb,amsfonts,fullpage}
\pagenumbering{gobble}
\begin{document}
	\begin{center}
		{\huge PJJ Tahap 3 IMO 2016 Tidak Resmi}
		
		\vspace{1em}
		
		{\Large Paket 4}
		
		\vspace{1em}
		
		DJ Ilhan rmx
	\end{center}
	\section{Aturan Main}
	\begin{itemize}
		\item Solusi dikumpul paling lambat \textbf{Sabtu, 16 April 2016 22:59 WIB}.
		\item Soal dibuat sehingga no. 1 kira-kira sesusah no. 2 IMO, no. 3 sesusah no. 3 IMO. No. 2 somewhere in between. Anak tahap 3 udah ga jaman ngerjain soal no. 1 atau 4.
		\item Solusi boleh diketik, discan ataupun difoto.
		\item Kirimkan ke: 7744han@gmail.com ATAU h.i@u.nus.edu ATAU e0008984@u.nus.edu. Tulis nama Anda di subjek email.
		\item Umpan balik akan diberikan ke Anda. (Udah kayak KTO aja)
		\item Juara 1, 2, dan 3 bakal dapat hadiah (woo!)
		\item Soal akan di-post di olimpiade.org setelah batas pengumpulan. Harap partisipasinya dalam meramekan forum anak bangsa.
		\item Above all, PJJ ini tidak di-enforce kok. Iseng-iseng aja.
	\end{itemize}
	\section{Soal-soal}
	(Balik lagi ke Bahasa Indonesia, supaya semangat kenegaraaan kalian tumbuh (?))
	\vspace{1em}
	\begin{enumerate}
		\item Diberikan $P$ dan $Q$ suku banyak monik dengan koefisien kompleks sehingga $P(P(x)) = Q(Q(x))$ untuk setiap $x$ kompleks. Buktikan $P(x) = Q(x)$ untuk setiap $x$ kompleks.
		\item Diberikan dua bilangan bulat positif $m$ dan $n$. Diketahui pula terdapat sebuah perkumpulan yang berisi banyak orang yang mana untuk setiap 2 orang $A$ dan $B$ di perkumpulan tersebut, antara $A$ dan $B$ saling mengenal atau keduanya saling tidak mengenal. Untuk setiap $k$ orang di perkumpulan tersebut (tentu saja banyak orang di perkumpulan tersebut $\ge k$), diketahui antara ada $2m$ orang $a_1, a_2, \dotsc, a_m, b_1, b_2, \dotsc, b_m$ sehingga $a_i$ dan $b_i$ saling mengenal untuk setiap $i$, atau ada $2n$ orang $c_1, c_2, \dotsc, c_n, d_1, d_2, \dotsc, d_n$ sehingga $c_i$ dan $d_i$ saling tidak mengenal untuk setiap $i$. Tentukan nilai $k$ terkecil dalam $m$ dan $n$.
		\item Cari semua $f: \mathbb{N} \to \mathbb{N}$ sehingga $(f(x) + y)(f(y) + x)$ kuadrat sempurna untuk semua $x, y \in \mathbb{N}$.
	\end{enumerate}
	\begin{center}
		\textit{``Saya sudah punya KTP loh'' - Gede Bagus Bayu Pentium}
	\end{center}
\end{document}