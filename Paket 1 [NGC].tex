\documentclass{article}
\usepackage{CJK}
\usepackage{amsmath,amsthm,amssymb,amsfonts,fullpage}
\pagenumbering{gobble}
\begin{document}
	\begin{center}
		{\huge PJJ Tahap 3 IMO 2016 Tidak Resmi}
		
		\vspace{1em}
		
		{\Large Paket 1}
		
		\vspace{1em}
		
		DJ Ilhan rmx
	\end{center}
	\section{Aturan Main}
	\begin{itemize}
		\item Solusi dikumpul paling lambat \textbf{Sabtu, 26 Maret 2016 23:59}.
		\item Soal dibuat sehingga no. 1 kira-kira sesusah no. 2 IMO, no. 3 sesusah no. 3 IMO. No. 2 somewhere in between. Anak tahap 3 udah ga jaman ngerjain soal no. 1 atau 4.
		\item Solusi boleh diketik, discan ataupun difoto.
		\item Kirimkan ke: 7744han@gmail.com ATAU h.i@u.nus.edu ATAU e0008984@u.nus.edu. Tulis nama Anda di subjek email.
		\item Umpan balik akan diberikan ke Anda. (Udah kayak KTO aja)
		\item Juara 1, 2, dan 3 bakal dapat hadiah (woo!)
		\item Soal akan di-post di olimpiade.org setelah batas pengumpulan. Harap partisipasinya dalam meramekan forum anak bangsa.
		\item Above all, PJJ ini tidak di-enforce kok. Iseng-iseng aja.
	\end{itemize}
	\begin{CJK}{UTF8}{gbsn}
		\section{问题}
 	\end{CJK}
	(Bahasa Mandarin, karena kalian mau ke Hong Kong)
	\vspace{1em}
	\begin{enumerate}
		\item Sebut sebuah bilangan asli \textit{tawas} bila setiap dua digit yang bersebelahan di representasi desimalnya berbeda paritas. Contohnya, 1, 12 dan 123454327 merupakan bilangan-bilangan tawas. Diberikan $n$ bilangan asli. Buktikan bahwa semua kelipatan dari $n$ tidak tawas jika dan hanya jika $n$ habis dibagi 20.
		\item Dua lingkaran $C_1$ dan $C_2$ berpotongan di $A$ dan $B$. Buat sebuah lingkaran $C_0$ yang berada di dalam $C_1$ dan $C_2$; lingkaran $C_0$ ini menyinggung $C_1$ di $D$ dan menyinggung $C_2$ di $E$. Misalkan garis $AB$ memotong $C_0$ salah satunya di $X$. Kemudian, $P$ adalah titik perpotongan $EX$ dengan $C_2$ dan $Q$ adalah titik perpotongan $FX$ dengan $C_1$. Jika $DE$ memotong $C_1$ dan $C_2$ masing-masing di $R$ dan $S$, serta $D$, $E$, $P$, $Q$, $R$, dan $S$ semuanya merupakan titik yang berbeda, buktikan bahwa $PQRS$ terletak pada satu lingkaran.
		\item Diberikan $n$ bilangan bulat positif. Diberikan bilangan-bilangan bulat $0 = a_0 < a_1 < \dotsb < a_n = 2n - 1$. Cari kardinalitas minimum himpunan $\{ a_i + a_j \, | \, 0 \le i \le j \le n \}$.
	\end{enumerate}
	\begin{center}
		\textit{``Saya bawa roket ke pelatihan'' - Rezky Arizaputra, 2015}
	\end{center}
\end{document}